\documentclass[11pt]{amsart}
\usepackage[margin=0.9in]{geometry}                % See geometry.pdf to learn the layout options. There are lots.
\geometry{letterpaper}                   % ... or a4paper or a5paper or ... 
%\geometry{landscape}                % Activate for for rotated page geometry
%\usepackage[parfill]{parskip}    % Activate to begin paragraphs with an empty line rather than an indent
\usepackage{graphicx}
\usepackage{amssymb}
\usepackage{epstopdf}
\usepackage{multicol}
\DeclareGraphicsRule{.tif}{png}{.png}{`convert #1 `dirname #1`/`basename #1 .tif`.png}

\usepackage{hyperref}
\hypersetup{
    colorlinks,
    citecolor=black,
    filecolor=black,
    linkcolor=black,
    urlcolor=black
}


\author{}
\begin{document}
\pagestyle{empty}
\setlength\parindent{0pt}

\Huge
\hspace*{\fill} \textbf{James Macek} \hspace*{\fill}
\normalsize
\vspace{1cm}

 \begin{tabular}{p{5cm} p{1cm} p{10cm}}
 \textbf{Address:}       &  & \textbf{Phone:} +1-647-458-9797                                                                                        \\
 Department of Economics &  & \textbf{Email:} \href{mailto:james.macek@mail.utoronto.ca}{james.macek@mail.utoronto.ca}                         \\
 University of Toronto   &  & \textbf{Website:} \href{https://jamesmacek.github.io}{jamesmacek.github.io}  \\
150 St.\ George St.       &  &                                                                                                                        \\
Toronto, Ontario         &  &                                                                                                                        \\
M5S 3G7, Canada          &  &                                                                                                                        
\end{tabular}

\line(1,0){470}


 \begin{tabular}{ p{4cm}  p{15cm}}
 \textbf{Citizenship:}           & Canadian                                     \\
                                 &                                              \\
\textbf{Research Interests:}     & Urban \& Real Estate economics,  International trade, Macroeconomics \\ 
                                 &                                           \\
\end{tabular}



\vspace{0.4cm}





\LARGE
\textsc{Education}
\vspace{0.2cm}

\normalsize
\begin{tabular}{ p{12.5cm}  p{5cm}}
  \large{Ph.D. in Economics, University of Toronto}  & 2025 (Expected) \\
   \multicolumn{2}{l}{ \hspace{.75cm}  \emph{Committee:}  Nathaniel Baum-Snow (supervisor), } \\
     \hspace{2.8cm}  William Strange, Kevin Lim, Joseph Steinberg  \\
 \hspace{2.90cm}   \\
  \\
  \large{M.A in Economics, University of Toronto}    & 2019            \\
   \\
  \large{Bachelor of Business Administration, University of Toronto Scarborough}   & 2018            \\
                                                     &                 
\end{tabular}

\vspace{0.4cm}


%Publications


\LARGE
\textsc{Research}
\vspace{0.2cm}
\normalsize

\begin{tabular}{ p{15.5cm}}
 \textbf{Lot Sizes, Welfare and Urban Structure: A view from the US} (Job Market Paper)                         \\

\\
 \textbf{Gentrification and Redevelopment in General Equilibrium}  with Guangbin Hong \\
\\
\textbf{Nonlinear Pricing in Housing Markets: Implications for Policy and Inequality} 
\normalsize
\end{tabular}

\vspace{0.4cm}

\LARGE
\textsc{Awards and Grants}
\vspace{0.2cm}


\normalsize 
\begin{tabular}{ p{12.5cm}  p{5cm}}
  Ontario Graduate Scholarship (\$5000 $\times$ 3) & 2024 \\
  AREUEA Travel Grant (\$1300)    & 2024         \\
  University of Toronto Doctoral Fellowship  (\$12,000 $ \times$ 5)  & 2019 - 2024  \\
  Yurgen Krumma Award in Economics & 2017
\end{tabular}

\vspace{0.4cm}

\LARGE
\textsc{Professional Experience}
\vspace{0.2cm}
\normalsize

\begin{tabular}{ p{12.5cm} p{5cm}}
Teaching Assistant & 2017 - present \\
\begin{itemize}
  \item ECO 204: Microeconomics for Commerce
  \item ECO 231: Economics of Global Trade
  \item ECO 380: Markets, Competition and Strategy
  \item ECO 220: Data Analysis and Applied Econometrics
\end{itemize} 
                   &                
\end{tabular}

\begin{tabular}{ p{12.5cm} p{5cm}}
                   &             \\
Research Assistant & 2018 - present \\
\begin{itemize}
  \item Nathaniel Baum-Snow: Advanced data analysis 
  \item Ambarish Chandra: data analysis
  \item Marco Gonzalez-Navarro \& World Bank DiME in Kigali, Rwanda (2017)
\end{itemize}      &             


\end{tabular}





\normalsize

\vspace{0.4cm}
\LARGE
\textsc{Conference Presentations}
\vspace{0.2cm}
\normalsize

\begin{tabular}{ p{12.5cm} p{5cm}}
Urban Economics Association Summer School & 2024 \\	
European Meeting of the Urban Economics Association (Student prize session) &				 2024 \\
Annual Conference of the Canadian Economics Association (Toronto)     & 2024 \\
University of British Columbia Sauder CUERE Symposium                   & 2024 \\
AREUEA Doctoral Poster Session & 2024 \\
\end{tabular}


\vspace{0.4cm}

\LARGE
\textsc{Refereeing Experience}
\vspace{0.2cm}
\normalsize

\begin{tabular}{ p{17.5cm} }
Journal of Urban Economics (x2)  \\
\end{tabular}
 

\vspace{0.4cm}
\LARGE
\textsc{Academic Service}
\vspace{0.2cm}

\normalsize
\begin{tabular}{ p{12.5cm} p{5cm}}
Co-Organizer of the Graduate Student Workshop  & 2023 - Present  \\
\end{tabular}

\vspace{0.4cm}
\LARGE
\textsc{Languages}
\vspace{0.2cm}

\normalsize
\begin{tabular}{ p{17.5cm}}
  English (native)  \\
  \emph{Programming:} R, Stata, Python, SQL, MATLAB, GIS (using R, Python and QGIS)
\end{tabular}


\vspace{0.4cm}
\LARGE
\textsc{References}
\vspace{0.2cm}

\normalsize
\begin{tabular}{ p{8cm} p{8cm}}
 Nathaniel Baum-Snow  &   William Strange            \\
Rotman School of Management   &  Rotman School of Management  \\
 University of Toronto     &  University of Toronto    \\
 105 St.\ George St.        & 105 St.\ George St.        \\
 Toronto, Ontario          & Toronto, Ontario          \\
 M5S 3E6, Canada           & M5S 3E6, Canada           \\
 Nate.Baum.Snow@rotman.utoronto.ca     & William.strange@rotman.utoronto.ca      \\
					        &           \\
                           &                           \\
  Kevin Lim			       &   Joseph Steinberg                       \\
  Department of Economics  &  Department of Economics                         \\
 University of Toronto     &       University of Toronto                    \\
 150 St.\ George St.        &       150 St.\ George St                     \\
 Toronto, Ontario          &      Toronto, Ontario                     \\
 M5S 3G7, Canada           &        M5S 3G7, Canada                    \\
kvn.lim@utoronto.ca       &       joseph.steinberg@utoronto.ca                    \\
					         &                           
\end{tabular}



%Refereed For


\vspace{1.5cm}
\begin{center}
\tiny Last Updated: \today
\end{center}

\newpage
\begin{center}
\LARGE
\textbf{Abstracts}
\normalsize
\end{center}
\line(1,0){505}

\begin{center}
\LARGE
\textbf{Housing Regulation and Neighborhood Sorting across the United States}\\
\large
(Job Market Paper)
\normalsize
\end{center}

In this paper, I consider the effect of minimum lot size regulation on welfare and urban structure. I show that minimal lots are the most expensive in the low-density neighborhoods of productive cities, and this can explain the sorting on income into these cities and neighborhoods. Motivated by this evidence, I construct a general equilibrium model in which households of heterogeneous incomes choose cities and neighborhoods, value affluent neighbors, and are burdened differently by regulation. A counterfactual deregulation exercise shows significant and progressive gains for renters that may offset the losses to landowners. The exercise also reveals two surprising results. First, any productivity gains that occur from the expansion of productive cities is largely nullified by the out-migration of affluent households who prefer regulated neighborhoods. Second, the neighborhood choice externality arising from the demand for affluent neighbors matters little for the average household, but has important distributional consequences. These results suggest that the most important consequence of deregulating housing markets is increasing housing affordability.  \\


\bigskip
\begin{center}
\LARGE
\textbf{Gentrification and Redevelopment in General Equilibrium}\\
\large
  with Guangbin Hong
\normalsize
\end{center}
The age of the housing stock affects sorting of high and low-income households into different neighborhoods within a city (Rosenthal, 2009). As neighborhoods undergo development and redevelopment over time, the spatial distribution of different types of households change considerably. There has been a heated debate on how to regulate housing redevelopment. On one hand, redeveloping old neighborhoods are expected to increase (high-quality) housing supply and decrease prices. On the other hand, such redevelopment also incurs significant gentrification and displacement of incumbent residents. To study the welfare consequences of redevelopment, we build a general equilibrium model that features forward-looking housing developers, heterogenous households with non-homothetic demand for housing, and costly movement across neighborhoods. Developers choose when to (re)develop, how many housing units to build, and the quality of housing units. Housing quality depreciates over time, which prompts household movements. We aim to quantify the impact of various housing polices that are designed to restrict or encourage redevelopment. 


\bigskip
\begin{center}
\LARGE
\textbf{Nonlinear Pricing in Housing Markets: Implications for Policy and Inequality }\\
\large
\normalsize
\end{center}
US housing prices have been rising rapidly in the past 40 years, to which there is a strong understanding of its consequences on the welfare of the average household. However, little is known about how prices have evolved differently for both low and high quality housing, and how this has differed across local markets. Many housing regulations naturally cause low quality housing to be relatively more expensive. Motivated by this idea, I propose a novel identification strategy to measure the causal effect of rising regulatory stringency on housing prices for each quality segment.  \\

\end{document}  